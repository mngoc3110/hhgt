\documentclass[12pt,a4]{article}
\usepackage{lipsum}
% Packages for font and Vietnamese support
\usepackage[utf8]{inputenc}
\usepackage[utf8]{vietnam}
\usepackage[vietnamese]{babel}
\usepackage{fontspec}  % For using system fonts

% Page geometry
\usepackage[paperheight=29.7cm,paperwidth=21cm,right=2cm,left=2cm,top=2cm,bottom=2.5cm]{geometry}
\usepackage[fontsize=15pt]{scrextend}

% Packages for math and graphics
\usepackage{amsmath,amsxtra,amssymb}
\usepackage{graphicx}
\usepackage{tikz}
\usetikzlibrary{angles, decorations.shapes, arrows.meta, decorations.markings, decorations.footprints, shapes, shadings, decorations, arrows, decorations.pathmorphing, calc, fadings, shapes.geometric, shapes.misc, shadows, decorations.text, positioning, decorations.fractals, scopes, backgrounds}

% Color and box packages
\usepackage[table,dvipsnames,hyperref]{xcolor}
\usepackage{vntcbox}
\usepackage[most]{tcolorbox}
\usepackage{varwidth}
\usepackage{fontawesome5}
\usepackage{pifont}
\usepackage[tikz]{bclogo}
\usepackage{cancel}
\usepackage{parnotes}
\usepackage{setspace}
% Header, title, and float settings
\usepackage{titlesec}
\usepackage{fancyhdr}
\usepackage{float}
\usepackage{hyperref}  % For hyperlinks

\titleformat{\section}
  {\normalsize\normalsize\bfseries} % Chỉnh kích thước chữ cho tiêu đề
  {\thesection}{1em}{} % Chỉnh số thứ tự của section
\renewcommand{\theenumi}{[\arabic{enumi}]} % Change format of numbers
\renewcommand{\thesection}{\small\arabic{section}}

\hypersetup{
    colorlinks=true,
    linkcolor=black,
    urlcolor=blue,
    pdfauthor={Tác giả},
    pdftitle={Tiêu đề tài liệu}
}


% Đặt khoảng trắng bên trái cho subsection
\titleformat{\subsection}[runin]
  {\normalfont\bfseries} % Font và kiểu chữ
  {}                     % Bỏ số mục
  {1em}                  % Khoảng cách giữa số và tiêu đề
  {\hspace{0em}}        % Khoảng thụt vào bên trái
  \usepackage{titlesec}
  \titlespacing*{\subsection}{0pt}{1em}{1.5em} 
\pagestyle{fancy}
\fancyhf{}
\fancyhead[L]{\textcolor[rgb]{0.168, 0.396,0.565}{\textbf{Hình học giải tích}}}
\fancyhead[R]{\textcolor[rgb]{0.168, 0.396,0.565}{\textbf{Bài 3 - chương 2}}}
% Đặt nội dung cho footer
\fancyfoot[C]{\rule{\textwidth}{0.4pt}\\[1ex] % Thanh ngang
\textcolor[rgb]{0.168, 0.396,0.565}{\textbf{Trang \thepage}} \\ % Số trang
} % Ngày

\begin{document}
\begin{titlepage}
  \begin{tikzpicture}[overlay,remember picture]
      %khung lớn màu xanh
      \draw[line width=5pt,color={rgb:red,0;green,0;blue,130}]  
      ($ (current page.north west) + (2.0cm, -2.0cm) $)
      rectangle
      ($ (current page.south east) + (-2.0cm, 1.6cm) $);
      %khung nhỏ
      \draw[line width=1pt]
      ($ (current page.north west) + (2.3cm, -2.3cm) $)
      rectangle
      ($ (current page.south east) + (-2.3cm, 1.9cm) $);
      %khung hình vuông nhỏ(4 khung)
      %Bên trái phía trên
      \draw[line width=1pt]
      ($ (current page.north west) + (1.6cm, -1.6cm) $)
      rectangle
      ($ (current page.south east) + (-18.7cm, 27.4cm) $);
      %phía trên bên phải
      \draw[line width=1pt]
      ($ (current page.north west) + (19.4cm, -1.6cm) $)
      rectangle
      ($ (current page.south east) + (-2.3cm, 27.4cm) $);
      %phía dưới bên trái
      \draw[line width=1pt]
      ($ (current page.north west) + (1.6cm, -27.8cm) $)
      rectangle
      ($ (current page.south east) + (-18.7cm, 1.2cm) $);
      %phía dưới bên phải 
      \draw[line width=1pt]
      ($ (current page.north west) + (19.4cm, -27.8cm) $)
      rectangle
      ($ (current page.south east) + (-2.3cm, 1.2cm) $);
      %các đường trang trí
      \draw[line width=0.7pt] (-0.05, 1) -- (16, 1);
      \draw[line width=0.7pt] (-0.05, -26.4) -- (16, -26.4);
      \draw[line width=0.7pt] (-1.2, -0.15) -- (-1.2, -25.25);
      \draw[line width=0.7pt] (17.15, -0.15) -- (17.15, -25.25);

      \draw[line width=0.7pt] (-1.2, -0.15) -- (-0.05, -0.15);
      \draw[line width=0.7pt] (-0.05, 1.01) -- (-0.05, -0.16);

      \draw[line width=0.7pt] (17.162, -0.15) -- (16, -0.15);
      \draw[line width=0.7pt] (16, 1.012) -- (16, -0.16);

      \draw[line width=0.7pt] (-1.21, -25.25) -- (-0.05, -25.25 );
      \draw[line width=0.7pt] (-0.05, -25.25) -- (-0.05, -26.41);

      \draw[line width=0.7pt] (17.162, -25.25) -- (16, -25.25 );
      \draw[line width=0.7pt] (16, -25.24) -- (16, -26.41);
  \end{tikzpicture}
 
  %nội dung trong khung
  \begin{center}
      TRƯỜNG ĐẠI HỌC SƯ PHẠM THÀNH PHỐ HỒ CHÍ MINH\\
      \vspace{0.2cm}
      KHOA TOÁN - TIN HỌC\\
      \begin{tikzpicture}
          \draw[thick, line width=0.7pt] (0,0) -- (4.7,0);
          \draw[thick, line width=0.7pt] (5.3,0) -- (10,0);
          \draw (5,0) node {*};
      \end{tikzpicture}
  \end{center}
  \begin{tikzpicture}[remember picture, overlay]
      \node[anchor=north] at ($(current page.north) - (2.5cm, 7cm)$) {
          \includegraphics[height=3cm]{image/DHSP.png}
      };
      \node[anchor=north] at ($(current page.north) - (-3.5cm, 6cm)$) {
          \includegraphics[height=4cm]{image/khoa_toan.png}
      };
  \end{tikzpicture}
  
  \vspace{6cm}
  \begin{center}
      {\LARGE\textbf{Tiểu luận giữa kì}}\\
      \vspace{0.5cm}
      Hình học giải tích\\
      \vspace{7cm}
      Giảng viên hướng dẫn: TS.Cao Trần Tứ Hải\\
      \vspace{2cm}
      Thành phố Hồ Chí Minh, ngày 6 tháng 1 năm 2024
  \end{center}
  \newpage
  % Thiết lập tiêu đề "Mục lục" với màu xanh
\renewcommand\contentsname{\textcolor{blue}{\textbf{Mục lục}}}

% Tạo mục lục
\tableofcontents
\newpage
% Nội dung giả lập để mục lục hoạt động chính xác

\section{Danh sách thành viên nhóm 6}
\begin{tabular}{|c|l|c|l|}
    \hline
    \textbf{Số thứ tự} & \textbf{Họ và tên} & \textbf{Mã số sinh viên} & \textbf{Chức vụ} \\ \hline
    1 & Lê Trọng Chí & 50.01.101.007 & Nhóm trưởng \\ \hline
    2 & Nguyễn Lê Minh Ngọc & 48.01.103.056 & Thành viên \\ \hline
    3 & Phạm Gia Hân & 50.01.101.018 & Thành viên \\ \hline
    \end{tabular}
\newpage
\section{Nội dung phân công công việc}
\begin{tabular}{|c|l|l|}
    \hline
    \textbf{Số thứ tự} & \textbf{Họ và tên} & \textbf{Nội dung phân công} \\ \hline
    1 & Lê Trọng Chí & Bài tập 5 \\ \hline
    2 & Nguyễn Lê Minh Ngọc & Bài tập 7, biên soạn \LaTeX \\ \hline
    3 & Phạm Gia Hân & Bài tập 6 \\ \hline
    \end{tabular}
    \newpage
\section{Lời mở đầu}
phạm gia hân tự viết
\section{Khái niệm cơ bản về phương pháp toạ độ}
\subsection{Phương pháp toạ độ trong mặt phẳng}
\begin{itemize}
    \item \textbf{Mục tiêu anffine}\\
    Trong không gian cho điểm O và 2 vector $\vec{OI} = \vec{i}, \vec{OJ} = \vec{j}$ không cùng phương. Tập hợp gồm điểm O và hai vector $\vec{i},\vec{j}$ được gọi là hệ toạ độ \textbf{affine} trong mặt phẳng. Khi đó:\\
    \begin{enumerate}
        \item Đường thẳng $Ox$ đi qua điểm $O$ và điểm $I$ gọi là trục hoành, đường thẳng $Oy$ đi qua điểm $O$ và điểm $J$ gọi là trục tung.\\
        \item Điểm $O$ gọi là gốc toạ độ. Hệ toạ độ \textbf{affine} như vậy được ký hiệu là: $O\vec{i}\vec{j}$ hoặc $Oxy$.\\
        nhớ vẽ hình.
        \item Với mỗi vector $\vec{u}$ bất kỳ trong không gian, tồn tại duy nhất một bộ số $(x,y)$ sao cho:
        \[
            \vec{u} = x\vec{i} + y\vec{j}
        \]
        Khi đó, $(x,y)$ được gọi là toạ độ của vector $\vec{u}$, ký hiệu: $\vec{u}(x,y)$ hoặc $\vec{u} = (x,y)$.
        \item Với mỗi điểm $M$ bất kỳ trong không gian, gọi $(x,y)$ là toạ độ của vector $\overrightarrow{OM}$, nghĩa là:
        \[
            \overrightarrow{OM} = x\vec{i} + y\vec{j}
        \]
        Khi đó, $(x,y)$ cũng được gọi là toạ độ của điểm $M$, kí hiệu: $M(x,y)$ hoặc $M = (x,y)$.
        \item Cho điểm $M(x,y) \text{ và } M'(x',y')$ thì ta có:
        \[
            \overrightarrow{MM'} = (x' - x, y' - y)
        \]
    \end{enumerate} 
    Trong hệ toạ độ \textbf{anffine} $O\vec{i}\vec{j}$, cho 2 vector $\vec{u}(x_1,y_1) \text{ và } \vec{v}(x_2,y_2).$ Khi đó, ta có các tính chất cơ bản sau:\\
    \begin{enumerate}
        \item $\vec{u} + \vec{v} = (x_1 + x_2)\vec{i} + (y_1 + y_2)\vec{j} = (x_1 + x_2, y_1 + y_2)$
        \item $\vec{u} = \vec{v} \Leftrightarrow x_1\vec{i} + y_1\vec{j} = x_2\vec{i} + y_2\vec{j} \Leftrightarrow \begin{cases}
            x_1 = x_2\\
            y_1 = y_2
        \end{cases}$
        \item $\vec{u}$ cùng phương $\vec{v} \Leftrightarrow \vec{u} = t\vec{v}$\\
        $\Leftrightarrow x_1 \vec{i} + y_1\vec{j} = tx_2\vec{i} + ty_2\vec{j}$\\
        $\Leftrightarrow \begin{cases}
            x_1 = tx_2\\
            y_1 = ty_2\\
        \end{cases}$\\
        Nếu $t > 0$ thì $\vec{u},\vec{v}$ cùng hướng.\\
        Nếu $t < 0$ thì $\vec{u},\vec{v}$ ngược hướng.
    \end{enumerate}
    \textbf{Phép đổi mục tiêu}\\
    Trong không gian, cho 2 hệ toạ độ \textbf{anffine} $O\vec{i}\vec{j}$ và $O'\vec{i'}\vec{j'}$. Giả sử đối với hệ toạ độ $O\vec{i}\vec{j}$, điểm $O'$ có toạ độ $(a_0,b_0),$ $\vec{i} = (a_1,b_1),\vec{j} = (a_2,b_2).$ đối với một điểm $M$ bất kì, gọi $(,y)$ là toạ độ của $M$ đối với hệ $O\vec{i}\vec{j}$ là $M(x',y')$ đối với hệ $O'\vec{i'}\vec{j'}$. Ta tìm sự liên hệ giữa các số $,y$ và $x',y'$. Theo định nghĩa của toạ độ vector và toạ độ điểm ta có:
    \[
        \begin{cases}
            x = a_1x' + a_2 y' + a_0\\
            y = b_1x' + b_2 y' + b_0\\
        \end{cases}
    \]
    Viết dưới dạng ma trận:
    \[
        \begin{pmatrix}
            x\\
            y
        \end{pmatrix}
        =
        \begin{pmatrix}
            a_1 & a_2\\
            b_1 & b_2
        \end{pmatrix}
        \begin{pmatrix}
            x'\\
            y'
        \end{pmatrix}
        +
        \begin{pmatrix}
        a_0\\
        b_0
        \end{pmatrix}
    \]
    \textbf{Trường hợp đặc biệt: Phép tịnh tiến mục tiêu:}
    Với $(a_1, b_1) = (1,0)$ và $(a_2, b_2) = (0,1)$, đẳng thức trên trở thành:  
    \[
        \begin{pmatrix}
            x\\
            y
        \end{pmatrix}
        =
        \begin{pmatrix}
            1& 0\\
            0 & 1
        \end{pmatrix}
        \begin{pmatrix}
            x'\\
            y'
        \end{pmatrix}
        +
        \begin{pmatrix}
        a_0\\
        b_0
        \end{pmatrix}
    \]
    \[
    \Leftrightarrow \begin{cases}
            x = x' + a_0\\
            y = y' + b_0
        \end{cases}
    \]
    Đây là công thức chuyển trục phép tịnh tiến từ $O\vec{i}\vec{j}$ sang $O'\vec{i}\vec{j}$
    \item \textbf{Mục tiêu trực chuẩn}\\
    Hệ toạ độ trực chuẩn là hệ toạ độ \textbf{anffine} $O\vec{i}\vec{j}$
\end{itemize}
\subsection{Phương pháp tọa độ trong không gian}
\begin{itemize}
    \item Mục tiêu anffine\\
    \item Mục tiêu trực chuẩn\\
\end{itemize}
\subsection{Cách chọn mục tiêu}
\section{Các dạng bài tập}
\section{Bài tập làm thêm}
\section{Tài liệu tham khảo}
\end{titlepage}
\end{document}
